% \iffalse meta-comment
% !TEX program  = XeLaTeX
%<*internal>
\iffalse
%</internal>
%<*readme>
----------------------------------------------------------------
ElegantLaTeX --- Elegant and Beautiful LaTeX templates.
Author: Ethan Deng and Liam Huang
E-mail: elegantlatex2e@gmail.com
Released under the LaTeX Project Public License v1.3c or later
See http://www.latex-project.org/lppl.txt
----------------------------------------------------------------

Some text about the package: probably the same as the abstract.
%</readme>
%<*internal>
\fi
\def\nameofplainTeX{plain}
\ifx\fmtname\nameofplainTeX\else
  \expandafter\begingroup
\fi
%</internal>
%<*install>
\input docstrip.tex
\keepsilent
\askforoverwritefalse
\preamble
----------------------------------------------------------------
ElegantLaTeX --- Elegant and Beautiful LaTeX templates. 
Author: Ethan Deng and Liam Huang
Source: https://github.com/ElegantLaTeX
E-mail: elegantlatex2e@gmail.com
Released under the LaTeX Project Public License v1.3c or later
See http://www.latex-project.org/lppl.txt
----------------------------------------------------------------

\endpreamble
\postamble

Copyright (C) 2009 by ElegantLaTeX <elegantlatex2e@gmail.com>

This work may be distributed and/or modified under the
conditions of the LaTeX Project Public License (LPPL), either
version 1.3c of this license or (at your option) any later
version.  The latest version of this license is in the file:

http://www.latex-project.org/lppl.txt

This work is "maintained" (as per LPPL maintenance status) by
You.

This work consists of the file  elegantlatex.dtx
and the derived files elegantlatex.pdf,
                      elegantnote.cls,
                      elegantbook.cls and
                      elegantpaper.cls.

\endpostamble
\usedir{tex/latex/elegantlatex}
\generate{
  \file{elegantnote.cls}{\from{\jobname.dtx}{note}}
  \file{elegantbook.cls}{\from{\jobname.dtx}{book}}
  \file{elegantpaper.cls}{\from{\jobname.dtx}{paper}}
}
%</install>
%<install>\endbatchfile
%<*internal>
\usedir{source/latex/elegantlatex}
\generate{
  \file{\jobname.ins}{\from{\jobname.dtx}{install}}
}
\nopreamble\nopostamble
\usedir{doc/latex/elegantlatex}
\generate{
  \file{README.txt}{\from{\jobname.dtx}{readme}}
}
\ifx\fmtname\nameofplainTeX
  \expandafter\endbatchfile
\else
  \expandafter\endgroup
\fi
%</internal>

%<note|book|paper>\NeedsTeXFormat{LaTeX2e}
%<note>\ProvidesClass{elegantnote}[2020/08/26 v1.1 Elegant Note Template]
%<book>\ProvidesClass{elegantbook}[2020/08/26 v1.2 Elegant Book Template]
%<paper>\ProvidesClass{elegantpaper}[2020/08/26 v1.3 Elegant Paper Template]


%<*driver>
\ProvidesFile{\jobname.dtx}[2020/04/12 v3.1.1 ElegantLaTeX dtx]
\documentclass{ltxdoc}
\OnlyDescription
\usepackage[T1]{fontenc}
\usepackage{lmodern}
\usepackage{geometry}
\geometry{
  a4paper,
  top=25.4mm, bottom=25.4mm,
  left=25mm, right=25mm,
  headheight=2.5cm,
  headsep=4mm,
  footskip=12mm
}
\linespread{1.35}
\geometry{a4paper, lmargin=2in, rmargin=1in}
\usepackage{fontspec}
\usepackage{graphicx}
\usepackage{newtxtext}
\renewcommand{\ttdefault}{cmtt}
\usepackage[UTF8]{ctex}
\usepackage[numbered]{hypdoc}
\newcommand\email[1]{\href{mailto:#1}{\nolinkurl{#1}}}
\usepackage[shortlabels,inline]{enumitem}
\setlist{nolistsep}
\usepackage{xcolor}
\EnableCrossrefs
\CodelineIndex
\RecordChanges
\begin{document}
  \DocInput{\jobname.dtx}
  \PrintChanges
  \PrintIndex
\end{document}
%</driver>
% \fi
%\GetFileInfo{\jobname.dtx}
%
%
% \DoNotIndex{\xmulineto, \arabic, \newenvironments, \LaTeX,
% \textbf, \thechapter, \theexam, \rmfamily, \renewcommand, \ignorespacesafterend, \vspace, \par, \draw, \begin, \end, \\, }
% \title{\bfseries
%  Elegant\LaTeX{} --- Templates Guide Book \\ 
%  ElegantLaTeX 系列模板说明文档
% }
%\author{
%  Ethan Deng \thanks{E-mail: elegantlatex2e@gmail.com} \& Liam Huang 
%}
%\date{Released \filedate}
%
%\maketitle
%
%\changes{v1.0}{2009/10/06}{First public release}
%
%
%\section{Elegant\LaTeX{} 简介 / Elegant\LaTeX{} Introduction}
% Elegant\LaTeX{} 项目组致力于打造一系列美观、优雅、简便的模板方便用户使用。目前由
% \href{https://github.com/ElegantLaTeX/ElegantNote}{ElegantNote},
% \href{https://github.com/ElegantLaTeX/ElegantBook}{ElegantBook},
% \href{https://github.com/ElegantLaTeX/ElegantPaper}{ElegantPaper} 组成,
% 分别用于排版笔记,书籍和工作论文。强烈推荐使用最新正式版本!本文将介绍本模板的一些设置内容以及基本使用方法。
% 如果您有其他问题,建议或者意见,欢迎在 GitHub 上给我们提交 
% \href{https://github.com/ElegantLaTeX/ElegantBook/issues}{issues} 或者邮件联系我们。
%
%
% Elegant\LaTeX{} Program developers are intended to provide you beautiful, 
% elegant, user-friendly templates. Currently, the Elegant\LaTeX{} is composed of 
% \href{https://github.com/ElegantLaTeX/ElegantNote}{ElegantNote}, 
% \href{https://github.com/ElegantLaTeX/ElegantBook}{ElegantBook}, 
% \href{https://github.com/ElegantLaTeX/ElegantPaper}{ElegantPaper}, 
% designed for typesetting notes, books, and working papers respectively. 
% Latest releases are strongly recommended! This guide is aimed at briefly 
% introducing the 101 of this template. For any other question, suggestion or comment, 
% feel free to contact us on GitHub \href{https://github.com/ElegantLaTeX/ElegantBook/issues}{issues} 
% or email us at \email{elegantlatex2e@gmail.com}.
%
% 
% 我们的联系方式如下,建议加入用户 QQ 群提问,这样能更快获得准确的反馈,加群时请备注 \LaTeX{} 或者 Elegant\LaTeX{} 相关内容。
% \begin{itemize}
%  \item 官网:\href{https://elegantlatex.org/}{https://elegantlatex.org/}
%  \item GitHub 网址:\href{https://github.com/ElegantLaTeX/}{https://github.com/ElegantLaTeX/}
%  \item CTAN 地址:\href{https://ctan.org/pkg/elegantbook}{https://ctan.org/pkg/elegantbook}
%  \item 文档 Wiki:\href{https://github.com/ElegantLaTeX/ElegantBook/wiki}{https://github.com/ElegantLaTeX/ElegantBook/wiki}
%  \item 下载地址:\href{https://github.com/ElegantLaTeX/ElegantBook/releases}{正式发行版},\href{https://github.com/ElegantLaTeX/ElegantBook/archive/master.zip}{最新版}
%  \item 微博:Elegant\LaTeX{}
%  \item 微信公众号:Elegant\LaTeX{}
%  \item 用户 QQ 群:692108391 
%  \item 邮件:\email{elegantlatex2e@gmail.com}
% \end{itemize}
%
%
% Contact Infos:
% \begin{itemize}
%   \item Homepage: \href{https://elegantlatex.org/}{https://elegantlatex.org/}
%   \item GitHub: \href{https://github.com/ElegantLaTeX/}{https://github.com/ElegantLaTeX/}
%   \item CTAN: \href{https://ctan.org/pkg/elegantbook}{https://ctan.org/pkg/elegantbook}
%   \item Wiki: \href{https://github.com/ElegantLaTeX/ElegantBook/wiki}{https://github.com/ElegantLaTeX/ElegantBook/wiki}
%   \item Download: \href{https://github.com/ElegantLaTeX/ElegantBook/releases}{release}, \href{https://github.com/ElegantLaTeX/ElegantBook/archive/master.zip}{latest version}
%   \item Weibo: Elegant\LaTeX{}
%   \item Wechat: Elegant\LaTeX{}
%   \item QQ: 692108391
%   \item Email: \email{elegantlatex2e@gmail.com}
% \end{itemize}
%
% \section{使用说明}
%\DescribeMacro{\email}
% 模板定义了邮件命令 \cs{email}, 有一个必选项,\marg{url}.
% \begin{verbatim}
%  \email{elegantlatex2e@gmail.com}
% \end{verbatim}
%
%\subsection{语言选项}
%
%
%\DescribeMacro{scheme}
% 本模板内含三套语言环境 \verb|lang=cn|、\verb|lang=en| 以及 \verb|lang=it| 
% \footnote{由 \href{https://github.com/VincentMVV}{VincentMVV} 提供意大利语翻译,
% 具体的内容见 \href{https://github.com/ElegantLaTeX/ElegantBook/issues/85}{Italian translation}。},
% 改变语言环境会改变图表标题的引导词(图,表),文章结构词(比如目录,参考文献等),
% 以及定理环境中的引导词(比如定理,引理等)。不同语言模式的启用如下:
% \begin{verbatim}
%  \documentclass[cn]{elegantbook} 
%  \documentclass[lang=cn]{elegantbook}
% \end{verbatim}
%
% \DescribeMacro{color}
% 如果需要自定义颜色的话请选择 \verb|nocolor| 选项或者使用 \verb|color=none|,然后在导言区定义 structurecolor、main、second、third 颜色,具体方法如下:
% \begin{verbatim}
%  \definecolor{structurecolor}{RGB}{0,0,0}
%  \definecolor{main}{RGB}{70,70,70}    
%  \definecolor{second}{RGB}{115,45,2}    
%  \definecolor{third}{RGB}{0,80,80} 
% \end{verbatim}
%
% \begin{figure}[!htbp]
% \centering
% \includegraphics[width=0.6\textwidth]{cert.pdf}
% \end{figure}
%
%
%
%
% \begin{macro}{\mymacro}
% We define a trivial macro, |\mymacro|, to illustrate
% the use of the |macro| environment.
%    \begin{macrocode}
\newcommand{\mymacro}{This is
  a \LaTeX{} macro.}
%    \end{macrocode}
% \end{macro}
%
%
%
%    \begin{macrocode}
%<*note>
%    \end{macrocode}
%
%
%    \begin{macrocode}
\begin{macro}{\examplemacro}
  \newcommand*\examplemacro[2][]{%
    Some code here, probably
}
\end{macro}
%    \end{macrocode}
%
%    \begin{macrocode}
%</note>
%    \end{macrocode}
% 
%
% \DescribeEnv{YOURENV}
% Put description of |YOURENV| here.
%
%
%
%\begin{environment}{example}
% \makebox[0pt][r]{\makebox[4cm][l]{\textcolor{blue}{\texttt{\small elegantbook} / }}} 最初我们在 Book 上定义了 |example| 环境。
%    \begin{macrocode}
%<*book>
\newcounter{exam}[chapter]
\setcounter{exam}{0}
\renewcommand{\theexam}{\thechapter.\arabic{exam}}
\newenvironment{example}[1][]{
  \refstepcounter{exam}
  \par\noindent\textbf{\color{main}{\examplename} \theexam #1 }\rmfamily}{
  \par\ignorespacesafterend}
%</book>
%    \end{macrocode}
%\end{environment}
%
% \section{Implementation}
% \begin{environment}{YOURENV}
% Put explanation of |YOURENV|’s implementation here.
%    \begin{macrocode}
\newenvironment{YOURENV}{}{}
%    \end{macrocode}
% \end{environment}
%
%\StopEventually{
%  \PrintChanges
%}
%
%    \begin{macrocode}
\PassOptionsToPackage{no-math}{fontspec}
\RequirePackage{iftex}
\ifdefstring{\ELEGANT@lang}{cn}{
  \ifXeTeX
    \ifdefstring{\ELEGANT@chinesefont}{founder}{
      \RequirePackage[UTF8,scheme=plain,fontset=none]{ctex}
      \setCJKmainfont[BoldFont={FZHei-B01},ItalicFont={FZKai-Z03}]{FZShuSong-Z01}
      \setCJKsansfont[BoldFont={FZHei-B01},ItalicFont={FZHei-B01}]{FZHei-B01}
      \setCJKmonofont[BoldFont={FZHei-B01},ItalicFont={FZHei-B01}]{FZFangSong-Z02}
      \setCJKfamilyfont{zhsong}{FZShuSong-Z01}
      \setCJKfamilyfont{zhhei}{FZHei-B01}
      \setCJKfamilyfont{zhkai}{FZKai-Z03}
      \setCJKfamilyfont{zhfs}{FZFangSong-Z02}
      \newcommand*{\songti}{\CJKfamily{zhsong}}
      \newcommand*{\heiti}{\CJKfamily{zhhei}}
      \newcommand*{\kaishu}{\CJKfamily{zhkai}}
      \newcommand*{\fangsong}{\CJKfamily{zhfs}}}{\relax}
    
    \ifdefstring{\ELEGANT@chinesefont}{nofont}{
      \RequirePackage[UTF8,scheme=plain,fontset=none]{ctex}}{\relax}

    \ifdefstring{\ELEGANT@chinesefont}{ctexfont}{
      \RequirePackage[UTF8,scheme=plain]{ctex}}{\relax}
  \else
    \ifdefstring{\ELEGANT@chinesefont}{ctexfont}{
      \RequirePackage[UTF8,scheme=plain]{ctex}}{\relax}
  \fi
  \AfterEndPreamble{
    % \renewcommand{\itshape}{\kaishu}
    \setlength\parindent{2\ccwd}}
}{\relax}
%    \end{macrocode}
%
%\Finale